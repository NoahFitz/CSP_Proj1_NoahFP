\documentclass{article}
\title{Procedures}
\begin{document}
\section{Equations from Kepler's Laws}
\subsection{Determining Tangential Velocity Given r and Position Given $\theta$}
\label{VelocitySection}
The first step in modeling the orbits of Earth and Mars was to find their tangential velocity at a given radius from
the centre of their elliptical orbit. This was determined using Kepler's law of areas, and one of its resulting 
equations (equations (\ref{Kepler3}) and (\ref{dAandV})), both found in Classical Mechanics by Taylor sections 3.5 and 
8.6, and resulted in equation(\ref{Velocity}) - the equation for L (equation (\ref{L})) was taken from the Georgia State University website in which the eccentricities and semi-major axis values were provided. 
\begin{equation}
	\frac{dA}{dt} = \frac{1}{2}rv_{\theta}
	\label{Kepler3}
\end{equation}
\begin{equation}
	\frac{dA}{dt} = \frac{L}{2\mu}
	\label{dAandV}
\end{equation}
\begin{equation}
	L =\mu\sqrt{GMa(1-e^2)}
	\label{L}
\end{equation}
\begin{equation}
	v_{\theta} = \frac{\sqrt{GMa(1-e^2)}}{r}
	\label{Velocity}
\end{equation}
Equation (\ref{ellipse}) was formed through reference to Classical Mechanics by Taylor secton 8.6. 
Equation (\ref{Radius}) was taken from the Georigia State website. Equation (\ref{difference}) was based on common
knowledge. 
\begin{equation}
	\frac{x^2}{a^2}+\frac{y^2}{b^2} = 1
\end{equation}
\begin{equation}
	\frac{b}{a} = \sqrt{1-e^2}
\end{equation}
\begin{equation}
	\frac{x^2}{a^2}+\frac{y^2}{a^2(1-e^2)} = 1
	\label{ellipse}
\end{equation}
\begin{equation}
	r(\theta) = \frac{a(1-e^2)}{1+ecos(\theta)}
	\label{Radius}
\end{equation}
\begin{equation}
	d = \sqrt{(x_{2} - x_{1})^2-(y_{2}-y_{1})^2}, x = rcos(\theta), y = rsin(\theta)
	\label{difference}
\end{equation}
\subsection{Approximating Position at Time t}
In order to find an exact equation for position at time t, one would have to integrate a variation of Kepler's
third law. This was initially attempted, but then abandoned in favor of an approximation using the equations listed in
section \ref{VelocitySection}. First, functions were defined for equations (\ref{Velocity}) and (\ref{Radius}). Then, a funtion called DeltaT was used to approximate the time elapsed between two tenths of a degree. Next, a distance function was written which takes a inputed time and adds predetermined $\Delta$t's in a while-loop until it reaches the given time. As these chuncks of time are summed, the function also sums the degrees elapse; the sum of the degrees is the actual output of the function. 

The results of the distance funtion and the radius function are then fed into a function which converts the polar coordinates to cartesian coordinates, and then into a function which determines the distance between the planets using equation (\ref{difference}).
\section{}
\end{document}
